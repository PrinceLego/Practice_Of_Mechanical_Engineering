\documentclass[12pt]{article}       % 設定文件類型為 article,字體大小為 12pt
\usepackage[T1]{fontenc}            % 設定 T1 字型編碼,確保特殊字元的正確顯示
\usepackage{lmodern}                % 強制使用 Latin Modern 字型,提高可讀性和相容性
\usepackage{fontspec}               % 允許使用 OpenType 和 TrueType 字型
\usepackage{graphicx}               % 支援插入圖片
\usepackage{amsmath}                % 提供數學環境和公式支持
\usepackage{csquotes}               % 提供引用格式支援
\usepackage{comment}                % 提供多行註解
\usepackage{ragged2e}
\usepackage{float}

%biber Practice_of_Mechanical_Engineering

%=================================================={{{參考文獻設定}}}==================================================

\usepackage[style=ieee, maxnames=99]{biblatex}          % 設定參考文獻格式為 IEEE,最多顯示 99 個作者
\addbibresource{Practice_of_Mechanical_Engineering.bib}                       % 添加參考文獻檔案 references.bib
\renewcommand{\bibfont}{\fontspec{Times New Roman}}     % 設定參考文獻字體為 Times New Roman
\renewcommand{\UrlFont}{\fontspec{Times New Roman}}     % 設定 URL 連結字體為 Times New Roman
\DeclareFieldFormat{url}{\url{#1}}                      % 格式化 URL                          % 引用你的 .bib 文件

%=================================================={{{目錄設定}}}==================================================

\usepackage{tocloft} % 自訂目錄格式

% 設定目錄的點線填充樣式
\renewcommand{\cftsecleader}{\cftdotfill{\cftdotsep}}           % 章節(section)
\renewcommand{\cftsubsecleader}{\cftdotfill{\cftdotsep}}        % 小節(subsection)
\renewcommand{\cftsubsubsecleader}{\cftdotfill{\cftdotsep}}     % 子小節(subsubsection)

% 設定圖目錄與表目錄的點線
\renewcommand{\cftdotsep}{1}  % 設定點的間距,使其在所有目錄(含圖、表)中都有效

% 設定目錄標題格式,使目錄、圖目錄、表目錄標題一致
\renewcommand{\contentsname}{\centering \LARGE \textbf{目錄}}    % 目錄標題置中,加粗
\renewcommand{\listfigurename}{\centering \LARGE \textbf{圖目錄}} % 圖目錄標題置中,加粗
\renewcommand{\listtablename}{\centering \LARGE \textbf{表目錄}} % 表目錄標題置中,加粗


%=================================================={{{字體設定}}}==================================================

% 設定英文字體
\newfontface\englishfont{Times New Roman}               % 自訂英文字體命令 \englishfont,使用 Times New Roman

\setmainfont[
    ItalicFont={Times New Roman Italic},                % 設定斜體
    BoldFont={Times New Roman Bold},                    % 設定粗體
    BoldItalicFont={Times New Roman Bold Italic}        % 設定粗斜體
]{Times New Roman}                                      % 設定主要英文字體為 Times New Roman

% 設定中文字體

\usepackage{xeCJK}                                      % 使用 xeCJK 宏包以支援中文
\renewcommand{\figurename}{圖}                           % 設定圖表名稱
\renewcommand{\tablename}{表}                            % 修改表格標題為「表」
\setCJKmainfont[BoldFont={標楷體-繁}, ItalicFont={標楷體-繁}] {標楷體-繁}

%=================================================={{{版面設定}}}==================================================

% 設定頁面邊界,適用 A4 紙張
\usepackage[top=2.54cm, bottom=2.54cm, left=3.18cm, right=3.18cm, a4paper]{geometry}

% 設定行距與段落格式
\usepackage{setspace}
\onehalfspacing % 設定 1.5 倍行距
\setlength{\parskip}{6pt} % 設定段落間距 6pt
\setlength{\parindent}{2em} % 設定段落首行縮排 2 個字元

%=============================================================================================================================
%=============================================================================================================================
%=============================================================================================================================

\begin{document}
%=================================================={{{封面}}}==================================================
\begin{titlepage}
    \centering
    \vspace*{1cm} % 增加上方間距

    {\LARGE \textbf{元智大學工程學院機械工程學系}} \\[0.5cm] % 標題較大且加粗
    {\LARGE {Department of Mechanical Engineering}} \\[0.5cm] % 標題較大且加粗
    {\LARGE {College of Engineering}} \\[0.5cm]
    {\LARGE {Yuan Ze University}}

    \vfill % 這一行讓前面的資訊靠上排列

    {\LARGE{孤輪阿罵的飆速輪椅}} \\[0.5cm]% 這行會上下左右完全置中
    {\LARGE{機械工程實務期末報告}} % 這行會上下左右完全置中

    \vfill % 這一行讓後面的資訊靠下排列

    {\LARGE {1100826 王子晨}}\\[0.5cm]
    {\LARGE {1100854 蘇威全}}\\[0.5cm]
    {\LARGE {1100862 施廷翰}}\\[0.5cm]
    {\LARGE {1100812 魏羽暘}}\\[0.5cm]
    {\LARGE {1100861 盧昀序}}\\[2.5cm]
    {\LARGE {指導教授:江右君、翁芳柏、余念一、吳昌暉}}\\[0.5cm]

\end{titlepage}
\newpage
%=================================================={{{封面}}}==================================================

\pagenumbering{roman}  
\setcounter{page}{1}  % 從 I 開始

%=================================================={{{中文摘要}}}==================================================

\section*{\centering 摘要}  % 只讓標題置中
\addcontentsline{toc}{section}{摘要}  % 手動加入摘要到目錄

%==============================摘要內容==============================

\hspace{2em}

\vspace{1.5em}
\noindent 關鍵字:
%==============================摘要內容==============================
\newpage  % 插入換頁命令,將目錄和後續內容分開

%=================================================={{{英文摘要}}}==================================================
\section*{\centering Abstract}  % 只讓標題置中
\addcontentsline{toc}{section}{Abstract}  % 手動加入摘要到目錄
%==============================摘要內容==============================
\hspace{2em}

\vspace{1.5em}
\noindent Keyword: 
%==============================摘要內容==============================
\newpage  % 插入換頁命令,將目錄和後續內容分開

%=================================================={{{目錄}}}==================================================

\begin{center}
    \tableofcontents    % 生成目錄
%========================={{{可有可無}}}=========================
    \newpage 

    \addtocontents{toc}{\protect\setcounter{tocdepth}{0}} % 暫時關閉目錄深度,讓圖目錄不顯示在目錄中
    \listoffigures      % 生成圖目錄
    \addtocontents{toc}{\protect\setcounter{tocdepth}{2}} % 恢復目錄深度(如果你的章節結構需要更深層級,請調整數值)
    \newpage  
    \listoftables       % 生成表目錄

%========================={{{可有可無}}}=========================
\end{center} 
%=================================================={{{內容開始}}}==================================================
\newpage  % 插入換頁命令,將目錄和後續內容分開
\pagenumbering{arabic}  % 開始使用阿拉伯數字頁碼
\setcounter{page}{1}  % 設定頁碼從 1 開始

%\englishfont{this is an example of mixed English and Chinese.}

\section{\centering 緒論}
%==============================內文==============================
\hspace{2em}

%==============================內文==============================

\subsection{問題敘述} 
%==============================內文==============================
\hspace{2em}

%==============================內文==============================

\subsection{設計規範} 
%==============================內文==============================
\hspace{2em}

%==============================內文==============================

\subsection{任務分工} 
%==============================內文==============================


\begin{itemize}
    \item 1100826 王子晨:演算法設計、期末報告撰寫
    \item 1100854 蘇威全:風扇設計
    \item 1100862 施廷翰:車體設計
    \item 1100812 魏羽暘:簡報製作
    \item 1100861 盧昀序:
\end{itemize}

%==============================內文==============================

\section{\centering 文獻回顧}
%==============================內文==============================
\hspace{2em}

%==============================內文==============================

\section{\centering 設計理念}

\subsection{葉片設計} 
%==============================內文==============================
\hspace{2em}

%==============================內文==============================

\subsection{電機與轉速} 
%==============================內文==============================
\hspace{2em}

%==============================內文==============================

\subsection{風道與結構設計} 
%==============================內文==============================
\hspace{2em}

%==============================內文==============================

\subsection{空氣動力學設計} 
%==============================內文==============================
\hspace{2em}

%==============================內文==============================

\subsection{車體設計} 
%==============================內文==============================
\hspace{2em}

%==============================內文==============================

\subsection{演算法設計} 
%==============================內文==============================
\hspace{2em}

%==============================內文==============================

\section{\centering 模擬與分析}
%==============================內文==============================
\hspace{2em}

%==============================內文==============================

\section{\centering 製程與實作}
%==============================內文==============================
\hspace{2em}

%==============================內文==============================

\section{\centering 測試與驗證}
%==============================內文==============================
\hspace{2em}

%==============================內文==============================

\section{\centering 結論與反思}
%==============================內文==============================
\hspace{2em}


%==============================內文==============================




\section{\centering 參考文獻}
%==============================內文==============================
\vspace{-3.5em}  % 減少與上方內容的間距
\renewcommand{\refname}{}  % 去除 "References" 標題
%\printbibliography  % 列出參考文獻
%==============================內文==============================

\section{\centering 附錄}

\subsection{程式碼} 
%==============================內文==============================
\hspace{2em}

%==============================內文==============================

\subsection{物料清單與發票收據} 
%==============================內文==============================
\hspace{2em}

%==============================內文==============================


\end{document}

%=============================================================================================================================
%=============================================================================================================================
%=============================================================================================================================


\begin{comment}

\section{\centering 緒論}
\subsection{研究問題} 
\subsubsection{違規車輛偵測}
    
%==============================圖片==============================

\begin{figure}[H]
    \centering
    \includegraphics[width=0.7\textwidth]{截圖 2025-01-24 03.21.17.jpg}     %圖片檔案名稱
    \caption{這是圖片的標題}    %圖片檔案名稱
    \label{fig:example2}    %為圖片添加標籤
    %如\ref{fig:example1}所示
\end{figure}
    
%==============================數學公式==============================

\begin{align}
    a &= b + c \label{eq:1}
    \\
    d &= e - f \label{eq:2}
    %式\ref{eq:2}
\end{align}
    
%==============================表格==============================

\begin{table}[H]
    \caption{MSI GP76 Leopard規格}
    \vspace{12pt} % 增加空格
    \renewcommand{\arraystretch}{1.5} % 調整行距以垂直置中
    \centering
    \begin{tabular}{|c|c|}
        \hline
        \textbf{元件} & \textbf{規格}                 \\ \hline
        中央處理器       & Intel(R) Core(TM) i7-10870H \\ \hline
        記憶體         & DDR4 16GB                   \\ \hline
        硬碟          & 1TB SSD                     \\ \hline
        顯卡          & NVIDIA® GeForce® RTX 3060   \\ \hline
        作業系統        & Ubuntu 18.04                \\ \hline
        電池          & 4-Cell 65 Battery (Whr)     \\ \hline
    \end{tabular}
    \label{tab:MSI GP76 Leopard}
    %(具體規格詳見表\ref{tab:MSI GP76 Leopard})
\end{table}
    
\begin{table}[H]
    \centering
    \caption{C922 Pro Stream 規格}
    \vspace{6pt} % 增加空格
    \label{tab:C922 Pro}
    \begin{tabular}{ll}
        \toprule
        \textbf{項目} & \textbf{規格} \\
        \midrule
        解析度  & 1280$\times$720 (HD) \\
        幀率  & 30fps \\
        水平視角 & 70.42$^\circ$ \\
        垂直視角  & 43.3$^\circ$ \\
        \bottomrule
    \end{tabular}
\end{table}

%==============================條列==============================

%無序條列

\begin{itemize}
    \item 第一項
    \item 第二項
    \item 第三項
\end{itemize}

%有序條列

\begin{enumerate}
    \item 第一項
    \item 第二項
    \item 第三項
\end{enumerate}

    \end{comment}
        